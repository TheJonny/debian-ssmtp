\documentstyle{article}  
\title{The Bugs in sSMTP.}
\author{David Collier-Brown}
\date{\today}
\begin{document}
\maketitle

\section{Introduction.}
  This is a history of the criticisms of and bugs found in sSMTP,
primarily by the anonymous reviewers of comp.sources.reviewed.

\section{Criticisms}
\begin{itemize}
\item README and man page were insufficient -- rewritten.
\item Bad MANIFEST format -- changed to use Larry Wall's {\it makekit(1)}
\item BSD sockets weren't documented as a dependency -- documented now.
\item Patchlevel wrong in documentation
\item strings.h and other BSD-isms in code -- corrected.
\item Fails to honor ``-f user'' and ``-fuser'' -- UNCHANGED.  
	This feature isn't really guaranteed, and I elected to not
	support it.
\item Does not guarantee addition of required headers, assuming
	mailhub will apply them -- changed, alas.  I assumed the
	sendmail/zmailer behavior of adding RFC-required fields,
	but this assumption didn't hold water: I wasn't sending
	and RFC-compliant message, and PP spotted it.  So now
	I do.
\item Does not support -t -- documented better.  This is a
	fundamental (mis) feature of the program, which is a filter.
	Adding -t requires at least a tempfile and possibly a
	queue, which makes the program radically different.  I've
	been asked (by Mike Marques) to write the more complex program,
	and may do so at a later date.  It certainly won't be called
	sSMTP, though! (The s stands for both small and simple).
\item Installation replaces /usr/lib/sendmail without saving a copy -- corrected.
\item ROOT= set in Makefile -- corrected.
\item LACKING= ill-documented -- corrected.
\item Latex errors in ssmtp\_plm.tex -- corrected.
\item Man page doesn't document sendmail flags -- corrected.
\item Won't compile on Sys V before R4 (Sys V R 3.2 frogs badly on the 
	BSD-specific  signals TTIN/TTOU).
\item Contained ``\verb+^L+'' characters, generate bad shar file -- corrected.
\item Too hard to reconfigure after compilation (also spotted
	my Mike Marques) -- added mail.conf file.
\item If you don't give it a to-address, it doesn't send a MAIL FROM:$<>$
	(found by me, in testing) -- corrected.
\item I had a printf(fp,buffer) in the code, which had problems
	printing lines containing %'s (found by Pok Ng) -- corrected.
\item I had a missing fp in a printf (found by me) -- corrected.

\end{itemize}

\end{document}
