\documentstyle{article}  
\title{The PLM (Program Logic Manual) for sSMTP.}
\author{David Collier-Brown}
\date{Started 17 August 1991, printed \today}
\begin{document}
\maketitle

\section{Introduction.}
    sSMTP is one half of a set of minimum-functionality mail programs for
workstations which use a departmental server as their mailhub.  The other
is/was rSMTP, but most people use pop.

    sSMTP sends mail for the users of the workstation.  Period.  Full stop.
It doesn't forward mail, it doesn't do mailing lists, it just sends mail.
And if it can't it refuses with an explanation, thereby sending the mail
scurrying back to the sender.

   If you don't have a departmental or other main mailhub that you pick your
mail up from, sSMTP may not be a lot of use to you. ``Pick up'' can mean
NFS-mounting the servers disk, running pop or imap or using rSMTP.

\subsection{History.}
   When I started this, I was writing a standalone mail receiver for
people who didn't want to use Sendmail at all.  

  This is applicable to a wide range of machine types:
\begin{enumerate}
	\item stations which have peculiar sendmails that end
		up taking too much time to build variant .cf files for,
	\item stations which are anonymous mail clients of non-sun servers 
		(Sun has a ``concentrate mail handling on the server''
		option in its sendmail.)
	\item Suns which don't run Yellow Snow but have stock shared
		libraries (and therefor can't communicate to anyone not
		in their hosts files)
	\item stations whose vendors don't read host requirements RFCs
		and so don't provide DNS service
	\item stations who are getting a bit long in the tooth and
		never did have DNS
	\item stations whose owners don't care to administer mail, but
		want the service to be reliable\label{reliable}
	\item stations who have ``mysterious problems'' with mail
	\item stations in a department (or lab) where the mail is
		supposed to be to/from the domain, not individual machines.
\end{enumerate}

    There are some limitations: this program doesn't honor ``sendmail -t'',
so one of the two gnu mail readers won't work with it, nor will some
other mail user agents (mailtool, if memory serves).

    Believe it or not, this is nothing but a filter.  You do have to specify
some parameters, but it basically reads stdin and writes to some other
MTA's stdin in turn.
    The mail receiver is also a filter, run from inetd and using /bin/mail
-d for final delivery.  It seems to be mostly of historical interest, but
can be obtained by harassing the author by email.

\subsection{Organization.}
  The paper has the same structure as the program, starting with main()
and working its way down.
  A structure chart looks like this: 
\begin{verbatim}

                              +------+
                              | main |
                              +------+
                                  |
                                  +------------+
                                  |            |
                              +-------+   +-----------+
                              | sSMTP |   | doOptions |
                              +-------+   +-----------+
                                  |       | getConfig |
                                  |       +-----------+
                                  |
         +-------------+----------+-----+--------------+
         |             |                |              |
   +----------+  +------------+   +-------------+  +-----------------------+
   | getpwuid |  | openSocket |   | putToSmtp   |  | addInitialHeaders     |
   +----------+  +------------+   +-------------+  +-----------------------+
                                  | getFromSmtp |  | recordRequiredHeaders |
                                  +-------------+  +-----------------------+
                                                   | addRequiredHeaders    |
                                                   +-----------------------+
\end{verbatim}
  The simplicity of the structure is deliberate: smtp is simple, but
not simple-minded.  The isomorphisms help make verification easier.

\section{File sSMTP.c}
% \subsection{unnamed}

  MailHub is the machine that provides the pop service. It can be
an A-record or a CNAME, but not a lonely MX.
  I once considered supporting MX records, but it really doesn't add much:
the program would have to iterate trying connections at some cost in
complexity just to be able to try for a fallback to a heavily supported
departmental service.  I decided to go for simplicity, and discovered that
it pays off: the error report is unambiguous, the reaction is immediate, and
as a result, whenever one of the mailhubs is down, the users call us up on
the phone and tell us exactly what has happened.  Sort of like SNMP using
volunteers instead of programs.

 We usually name mailhubs ``<department>.yorku.ca'', so that people can use
finger to see if mail has been delivered/read. 

  If you don't have DNS, you do have to have your mailhub name defined in 
your /etc/hosts file or YP (now NIS) maps.  

  REWRITE\_DOMAIN is the domain that sSMTP will use in ``From:'' lines.  If
it is defined, sSMTP will do hostname-hiding.  The system-defined hostname,
with any domain it includes, will be used otherwise. 
  We use FQDNs as hostnames.  If you don't, you can set HostName in the /usr/local/etc/mail.conf
file, along with the mailhub, person to receive root's mail and the rewrite domain.

\subsection{function main()}
	This function does little save set signals, assign globals, 
parse arguments and pass non-options to ssmtp.

\subsection{function doOptions()}
	This makes the program look like /usr/lib/sendmail, by
interpreting the command-line options and doing one of three
things with them:

\begin{itemize}
\item Ignore them.  They have no particular importance in context.
\item Emulate them, so that this program can be treated just like
	sendmail by both humans and other programs.  For example,
	the ``mailq'' command is emulated by saying there is nothing 
	in queue.
\item Explain them. If an explanation is given, the program is unable
	to emulate the command and is about to exit with a non-zero
	return code.  For example, /usr/lib/sendmail -bd (run in daemon mode)
	is diagnosed as impossible, the program explains its inability,
	then exits.  This tends to catch or at least flag misuses.
\end{itemize}

\subsection{function getConfig()}
    This function reads the domain name, the person who gets ``root'' mail
and any other site-specific options from a configuration file.  

\subsection{function sSMTP()}

  This is the protocol machine.  All the policy lives here, as do the
response messages.  This makes arguing about the fidelity of the
implementation much easier, just like a recursive descent compiler.
  This is important: this program has to interwork with some fragile
software out there, not all of which was written at Berkeley.

  There are two error conditions which the program can not deal with:
impossibilities and fatal errors.  Both cause the program to exit with
a message and a non-zero return code.
  The impossible conditions are things like no host name for
the machine or no known user name for the uid running the program.
Merely fatal errors are an inability to open a socket to the
smtp port. That one means the mailhub is down and nothing can be done
about it locally.

  Whenever the program exits, the calling program should arrange to
do something sane, such as saving messages to dead.letter. Alas, few
if any do: several just delete the mail silently.  Therefor ssmtp
will write is input to $HOME/dead.letter if it encounters an impossibility
or a fatal error.  

\begin{description}
\item[The unnamed step.] 
	sSMTP must connect to the mailhub, or fail as above.
\item[The HELO step.]
	sSMTP sends a HELO, then listens for a 2XX return code.
	
	It elects to fail if smtp server refuses to accept our hostname.
	Normally this can't happen (according to the RFC) so this is 
	treated as an ``impossible condition''.  We agree, and handle it 
	the same way as other impossible: we stop processing, explain and
	exit with a non-zero code.  This is what we'd do if it were a 
	well-defined fatal error condition, mind you\ldots

\item[The MAIL From step] 
	sSMTP does a ``from'' with the hostname and userid of the sender.
	It fails properly if smtp says no or wait.

\item[The RCPT To step]
	sSMTP does one or more ``to''s, and fails unless all are accepted. 
	Mailhubs are required to accept all: if they don't, we're in deep
	trouble and its better to tell the end user than deliver some 
	messages and fail to deliver others.
	Both sendmail and zmailer accept all to addresses, so we're fairly
	safe with ``normal'' atomic actions. Don't try using may rSMTP on
	a mailhub, though: it's one of the few programs that will refuse
	individual addresses.

\item[The DATA step]
	sSMTP send data until eof on input.  This theoretically could
	cause a doubled ``.'' to appear at the end of the message if the
	users of ``/usr/lib/sendmail'' insert lonely dots themselves.  That
	has happened with some public-domain mail user agents. 	Harmless, but 
	unexpected.  And wrong.

	This is where the normal failure occurs: the server can't
	take the message for some good reason, says ``no way'', and
	sSMTP exits with a failure indication and tries to save the
	message to dead.letter.  Neither sendmail nor zmailer
	try to reject individual addresses before the DATA step, so
	this is where we see the aborts.

\item[The QUIT step] does a clean exit from the protocol machine.
	Ironically, this could fail. We ignore this possibility,
	as it appears lack a meaning, and reporting it as an error
	to the call could result bogus failures being reported to
	a human.

\end{description}

	sSMTP logs results on sucessfull completion, or for any
exit via ``die''.

  Finally, a timeout results in a lost connection message (for half-open
connections) and a log message, followed by an exit with a failure
indication and the usual save to dead.letter\ldots

\section{Supporting libraries.}
\subsection{I/O library}
	This library makes smtp sockets look like Unix-domain fd's.
Special thanks to Blair P. Houghton, who developed a socket opening
function I've adapted.
	The rest of the library is a trio of get functions: one
to deal with boolean-like results, its server which deals with
possibly multi-line responses, and a final server which makes 
reading from an network socket look like fgets on a stream.
	Plus a put function, for obvious purposes.

\subsection{Alarm library}
	An alarmHandler causes an exit to ``die'' on timeouts,
via a traditional but non-portable and ill-defined longjmp.

\subsection{Die and Flush functions}
	These functions try to deal with fatal and impossible
errors, by producing as meaningful a diagnostic as possible
and as useful a dump to dead.letter as can be managed.  Alas,
errors in error-exit routines themselves may not be as well handled
as others elsewhere: running out of disk space here is a problem.

\subsection{Logging}
	Logging is done effectively but inelegantly, via a cloud
of #ifdefs to log files, syslog and stderr.

\subsection{Configuration file library}
	These few functions read and interpret the configuration file,
/usr/local/etc/mail.conf (or /etc/mail.conf), by setting
the globals Root, MailHub, HostName and RewriteDomain from the contents
of the file.

\subsection{Rewriting/Standardization library}
	This small library looks after making sure that required
headers are generated, if the user agent fails.  Others are 
left up to the mailhub, as it's the ``canonical'' source of the
message.
	The one and only rewriting function lives here, too:
it rewrites entire from lines iff REWRITE\<_DOMAIN is defined.
I almost got away without it, but I can't guarantee I'd have
good mailhubs.  Some of the things I get stuck with I can't
really recommend (and I don't!)

\section{File Strings.c}
  This file is some string utilities, including ones known
to be absent on machines I've tested.

\end{document}
%
% $Header: /home/matt/src/Packages/sSMTP/ssmtp-2.38/RCS/ssmtp_plm.tex,v 1.1 2000/05/05 17:01:29 matt Exp $
